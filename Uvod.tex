\chapter{Uvod}

U današnje, digitalno doba, glazbene platforme poput Spotifya postale su dio svakodnevnog života ljubitelja glazbe. Spotify je platforma koja pruža ogroman katalog pjesama te sakuplja značajne količine podataka o korisničkim preferencijama i glazbenim trendovima. Analiza ovih podataka postaje ključna kako bismo bolje razumjeli obrasce ponašanja slušatelja, usmjeravali marketinške strategije, te optimizirali glazbene ponude.

Ovaj projekt usredotočit će se na eksploratornu analizu podataka vezanih uz glazbu na Spotifyu, s fokusom na skup podataka koji uključuje različite informacije o pjesmama i playlistama. Stupci poput "track\_name", "track\_artist", "track\_popularity" i mnogi drugi pružaju bitne informacije o karakteristikama pjesama.

Kroz analizu ovih podataka, istražit ćemo pitanja poput koje vrste glazbe dominira na određenim playlistama, kako se popularnost pjesama mijenja tijekom vremena, te kako određene glazbene karakteristike (npr., danceability, energy) utječu na ukupnu popularnost pjesme. Pritom ćemo razmotriti kako se zajednički elementi među najuspješnijim pjesmama na platformi mogu povezati s određenim glazbenim žanrovima.

Ovaj seminar pružit će uvid u kompleksnost podataka koji okružuju glazbene platforme poput Spotifya i istaknuti važnost eksploratorne analize u otkrivanju ključnih uzoraka i informacija koje mogu koristiti glazbenoj industriji, marketinškim stručnjacima i ljubiteljima glazbe diljem svijeta.


\eject



