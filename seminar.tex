%definira klasu dokumenta 
\documentclass[12pt]{report} 

%prostor izmedu naredbi \documentclass i \begin{document} se zove uvod. U njemu se nalaze naredbe koje se odnose na cijeli dokument
	
	%osnovni LaTex ne može riješiti sve probleme, pa se koriste različiti paketi koji olakšavaju izradu željenog dokumenta
	\usepackage[croatian]{babel} 
	\usepackage{amssymb}
	\usepackage{amsmath}
	\usepackage{txfonts}
	\usepackage{mathdots}
	\usepackage{titlesec}
	\usepackage{array}
	\usepackage{lastpage}
	\usepackage{etoolbox}
	\usepackage{tabularray}
	\usepackage{color, colortbl}
	\usepackage{adjustbox}
	\usepackage{geometry}
	\usepackage[classicReIm]{kpfonts}
	\usepackage{hyperref}
	\usepackage{fancyhdr}
	
	\usepackage{float}
	\usepackage{setspace}
	\restylefloat{table}
	\usepackage{listings}
	\usepackage{color}
	
	\definecolor{dkgreen}{rgb}{0,0.6,0}
	\definecolor{gray}{rgb}{0.5,0.5,0.5}
	\definecolor{mauve}{rgb}{0.58,0,0.82}
	
	\lstset{frame=tb,
		language=Java,
		aboveskip=3mm,
		belowskip=3mm,
		showstringspaces=false,
		columns=flexible,
		basicstyle={\small\ttfamily},
		numbers=none,
		numberstyle=\tiny\color{gray},
		keywordstyle=\color{blue},
		commentstyle=\color{dkgreen},
		stringstyle=\color{mauve},
		breaklines=true,
		breakatwhitespace=true,
		tabsize=3
	}
	
	\patchcmd{\chapter}{\thispagestyle{plain}}{\thispagestyle{fancy}}{}{} %redefiniranje stila stranice u paketu fancyhdr
	
	%oblik naslova poglavlja
	\titleformat{\chapter}{\normalfont\huge\bfseries}{\thechapter.}{20pt}{\Huge}
	\titlespacing{\chapter}{0pt}{0pt}{40pt}
	
	
	\linespread{1.3} %razmak između redaka
	
	\geometry{a4paper, left=1in, top=1in,}  %oblik stranice
	
	\hypersetup{ colorlinks, citecolor=black, filecolor=black, linkcolor=black,	urlcolor=black }   %izgled poveznice
	
	
	%prored smanjen između redaka u nabrajanjima i popisima
	\newenvironment{packed_enum}{
		\begin{enumerate}
			\setlength{\itemsep}{0pt}
			\setlength{\parskip}{0pt}
			\setlength{\parsep}{0pt}
		}{\end{enumerate}}
	
	\newenvironment{packed_item}{
		\begin{itemize}
			\setlength{\itemsep}{0pt}
			\setlength{\parskip}{0pt}
			\setlength{\parsep}{0pt}
		}{\end{itemize}}
	
	%boja za privatni i udaljeni kljuc u tablicama
	\definecolor{LightBlue}{rgb}{0.9,0.9,1}
	\definecolor{LightGreen}{rgb}{0.9,1,0.9}
	
	%Promjena teksta za dugačke tablice
	\DefTblrTemplate{contfoot-text}{normal}{Nastavljeno na idućoj stranici}
	\SetTblrTemplate{contfoot-text}{normal}
	\DefTblrTemplate{conthead-text}{normal}{(Nastavljeno)}
	\SetTblrTemplate{conthead-text}{normal}
	\DefTblrTemplate{middlehead,lasthead}{normal}{Nastavljeno od prethodne stranice}
	\SetTblrTemplate{middlehead,lasthead}{normal}
	
	%podesavanje zaglavlja i podnožja
	
	\pagestyle{fancy}
	\lhead{Osnove statističkog programiranja}
	\rhead{Spotify Songs}
	\lfoot{Petra Buršić, Diego Mišetić, }
	\cfoot{stranica \thepage/\pageref{LastPage}}
	\rfoot{\today}
	\renewcommand{\headrulewidth}{0.2pt}
	\renewcommand{\footrulewidth}{0.2pt}

\begin{document}
	\begin{titlepage}
		\begin{center}

	\vspace*{\stretch{1.0}} %u kombinaciji s ostalim \vspace naredbama definira razmak između redaka teksta
	\LARGE Osnove statističkog programiranja\\
	\large Ak. god. 2023./2024.\\
	
	\vspace*{\stretch{3.0}}
	
	\huge $$Spotify$$\\
	

		\end{center}	
	\end{titlepage}
\end{document}
