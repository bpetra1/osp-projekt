\chapter{Eksploratorna analiza}

\section{Opis atributa}
\begin{table}[h]
	\centering
	\begin{tabular}{|c|c|p{8cm}|}
	\hline
	\textbf{Atribut} & \textbf{Tip podatka} & \textbf{Opis} \\ \hline
	track\_id & character & Jedinstveni ID pjesme \\ \hline
	track\_name & character & Naziv pjesme \\ \hline
	track\_artist & character & Izvođač pjesme \\ \hline
	track\_popularity & double & Popularnost pjesme (0-100) \\ \hline
	track\_album\_id & character & Jedinstveni ID albuma \\ \hline
	track\_album\_name & character & Naziv albuma na kojem se nalazi pjesma \\ \hline
	track\_album\_release\_date & character & Datum izlaska albuma \\ \hline
	playlist\_name & character & Naziv playliste \\ \hline
	playlist\_id & character & Jedinstveni ID playliste \\ \hline
	playlist\_genre & character & Žanr playliste \\ \hline
	playlist\_subgenre & character & Podžanr playliste \\ \hline
	danceability & double & Plesnost (koliko je pjesma prikladna za plesanje u rasponu 0.0-1.0) \\ \hline
	energy & double & Energičnost (perceptualna mjera intenziteta i aktivnosti u rasponu 0.0-1.0) \\ \hline
	key & double & Ukupni tonalitet pjesme \\ \hline
	loudness & double & Glasnoća pjesme u decibelima \\ \hline
	mode & double & Modus pjesme (1 - veliki, 0 - mali) \\ \hline
	speechiness & double & Prisutnost izgovorenih riječi u pjesmi \\ \hline
	acousticness & double & Mjera povjerenja je li pjesma akustična u rasponu od 0.0 do 1.0 \\ \hline
	instrumentalness & double & Sadrži li pjesma vokale \\ \hline
	liveness & double & Detektira prisutnost publike u snimci \\ \hline
	valence & double & Mjera od 0.0 do 1.0 koja opisuje glazbenu pozitivnost koju prijenosi pjesma \\ \hline
	tempo & double & Ukupno procijenjeni tempo pjesme u udarcima po minuti (BPM) \\ \hline
	duration\_ms & double & Trajanje pjesme u milisekundama \\ \hline
	\end{tabular}
	\caption{Opis tablice podataka o glazbi}
	\label{tab:glazba}
	\end{table}
\section{Proces učitavanja i prilagodbe podataka}

	\subsubsection{Proces učitavanja podataka}
	
		\textbf{Učitavanje podataka:}
		\begin{figure}[H]
			\centering
			\includegraphics[scale=0.9]{slike/ucitavanje.png}
			% Veličina slike u odnosu na originalnu datoteku i pozicija slike
			\caption{\textbf{Učitavanje podataka u R-u}}
		\end{figure}
		
		U prikazanom kodu sa slike, koristimo različite R pakete kako bismo pripremili i istražili skup podataka "spotify\_songs.csv". Prvo, koristimo pakete poput \textbf{readr}, \textbf{dplyr} i \textbf{stringr} za čitanje i manipulaciju podacima. Nakon toga, prikazujemo prvih nekoliko redova podataka pomoću funkcije \textbf{head} kako bismo dobili inicijalni uvid u strukturu podataka.
		
		Zatim, koristimo funkciju \textbf{glimpse} za detaljniji pregled strukture podataka, prikazujući informacije o varijablama, njihovim tipovima podataka i prvim redovima podataka. Na kraju, koristimo funkciju \textbf{summary} kako bismo dobili osnovne statističke informacije o numeričkim varijablama u skupu podataka.
		
		Ovi koraci omogućuju nam osnovni uvid u strukturu podataka prije nego što nastavimo s daljnjom analizom i vizualizacijom.

	
	\subsubsection{Proces prilagodbe podataka}
	



\section{Vizualizacija Podataka}
	Vizualizacija podataka postaje ključna komponenta analize i interpretacije kompleksnih skupova podataka. 
	U ovom podpoglavlju istražujemo moć vizualizacije u kontekstu glazbene platforme Spotify, prezentirajući neke od grafova kako bismo bolje razumjeli glazbene obrasce, preferencije slušatelja te dinamiku glazbene industrije.
	
	\subsubsection{1) Histogram of song popularity}
	
	\textbf{Opis grafa:}
	
	Ovaj graf prikazuje histogram popularnosti. Prikazuje distribuciju popularnosti pjesama. Na x-osi nalaze se razine popularnosti pjesama, a y-osi broj pjesama koje se nalaze u pojedinoj razini popularnosti.
	Ovaj histogram omogućava vizualni pregled koje su razine popularnosti češće, a koje rjeđe. 

	\textbf{Slika grafa:}
	\begin{figure}[H]
		\includegraphics[scale=0.9]{slike/Histogram of song popularity.png}
		%veličina slike u odnosu na originalnu datoteku i pozicija slike
		\centering
		\caption{Histogram of song popularity}
		
	\end{figure}

	\subsubsection{2) Top 10 Artists Based on Popularity}
	
	\textbf{Opis grafa:}
	
	Ovaj graf prikazuje deset najpopularnijih glazbenih izvođača temeljem prosječne popularnosti njihovih pjesama. Izračunata je srednja vrijednost popularnosti za svakog izvođača, a zatim su odabrani najbolji deset izvođača prema toj mjeri popularnosti.
	
	Na x-osi su navedeni izvođači, poredani prema visini prosječne popularnosti, dok y-os prikazuje prosječnu popularnost. Svaki šareni stupac predstavlja jednog izvođača, a visina stupa označava njegovu prosječnu popularnost.
	
	Ovaj graf pruža brz i pregledan način usporedbe popularnosti izvođača, omogućujući identifikaciju najboljih deset temeljem prosjeka popularnosti njihovih pjesama.
	
	\textbf{Slika grafa:}
	\begin{figure}[H]
		\includegraphics[scale=0.9]{slike/Top 10 popularity}
		%veličina slike u odnosu na originalnu datoteku i pozicija slike
		\centering
		\caption{Top 10 Artists Based on Popularity}
		
	\end{figure}


	\subsubsection{3) Average popularity of songs per year}
	
	\textbf{Opis grafa:}
	
	Ovaj stupčasti graf prikazuje prosječnu popularnost pjesama po godinama u razdoblju od 2000. godine do 2020. godine. X-os ovog grafa su godine u navedenom razdoblju (svaki stupac predstavlja jednu godinu), dok y-os predstavlja prosječnu popularnost. 
	Uvidom u ovaj graf možemo jednostavno vidjeti u kojoj su godini pjesme imale najveću popularnost, te vidjeti kako se popularnost mijenjala tokom tih 20 godina.

	
	\textbf{Slika grafa:}
	\begin{figure}[H]
		\includegraphics[scale=0.9]{slike/Average popularity of songs per year.png}
		%veličina slike u odnosu na originalnu datoteku i pozicija slike
		\centering
		\caption{Average popularity of songs per year}
		
	\end{figure}
	
		\subsubsection{4) Energy Distribution Across Playlist Genre}
	
	\textbf{Opis grafa:}
	
	Ovaj graf prikazuje distribuciju energije (y-os) na temelju različitih žanrova playlista (x-os). Svaki boxplot predstavlja jedan žanr, a njegova visina odražava raspon energije unutar tog žanra. Unutar svakog boxplota nalazi se pravokutnik koji predstavlja interkvartilni raspon, a linija unutar pravokutnika označava medijan energije.
	
	Dodatno, postojanje "notcha" u sredini svakog boxplota pruža informaciju o razlikama u medijanima između žanrova.
	
	
	\textbf{Slika grafa:}
	\begin{figure}[H]
		\includegraphics[scale=0.9]{slike/Genre-Energy.png}
		%veličina slike u odnosu na originalnu datoteku i pozicija slike
		\centering
		\caption{Energy Distribution Across Playlist Genre}
		
	\end{figure}
	
		\subsubsection{5) Distribution of Genres and Subgenres}
	
	\textbf{Opis grafa:}
	
		Ovaj graf prikazuje broj playlista unutar određenih glavnih žanrova, razdijeljenih prema podžanrovima. Na x-osi su navedeni glavni žanrovi playlista, dok y-os pokazuje broj playlista. Svaki šareni segment na stupcu predstavlja određeni podžanr unutar glavnog žanra.
		
		Stupci su složeni jedan na drugi kako bi se vizualno prikazala distribucija podžanrova u okviru svakog glavnog žanra. 
	
	
	\textbf{Slika grafa:}
	\begin{figure}[H]
		\includegraphics[scale=0.9]{slike/Genre-Subgenre.png}
		%veličina slike u odnosu na originalnu datoteku i pozicija slike
		\centering
		\caption{Distribution of Genres and Subgenres}
		
	\end{figure}


		\subsubsection{6) The relationship between energy and valence by genre}
    
    \textbf{Opis grafa:}
    
    	Ovaj graf prikazuje odnos između energije i valencije. N a x-osi nalazi se energija koja može biti u rasponu između 0 i 1, a na y-osi nalazi se valencija koja može biti u isto rasponu kao i energija. Svaka točka na grafu prikazuje jednu pjesmu , a njezina pozicija prikazuje odnos energija-valencija. Svaka boja točke prikazuje različiti žanr. 
    
    \textbf{Slika grafa:}
    \begin{figure}[H]
        \includegraphics[scale=0.9]{slike/The relationship between energy and valence by genre.png}
        %veličina slike u odnosu na originalnu datoteku i pozicija slike
        \centering
        \caption{The relationship between energy ad valence by genre}
        
    \end{figure}

	
		\subsubsection{7) Danceability and Energy with Popularity}
	
	\textbf{Opis grafa:}
	
	Ovaj šareni graf prikazuje odnos između plesnosti (x-os) i energije (y-os) za različite glazbene pjesme. Svaka točka na grafu predstavlja pojedinu pjesmu, a njezina boja označava razinu popularnosti. Tamnije crvene nijanse označavaju popularnije pjesme, dok svjetlije plave nijanse ukazuju na manju popularnost.
	
	Graf pruža uvid u raznolikost glazbenih preferencija te naglašava da glazbene osobitosti kao što su plesnost i energija nisu nužno ključni faktori koji određuju popularnost pjesama na temelju analize ovog skupa podataka.
	
	\textbf{Slika grafa:}
	\begin{figure}[H]
		\includegraphics[scale=0.9]{slike/Dance-Energy-popularity.png}
		%veličina slike u odnosu na originalnu datoteku i pozicija slike
		\centering
		\caption{ Danceability and Energy with Popularity}
		
	\end{figure}


	\subsubsection{8) Histogram of song durations}
    
    \textbf{Opis grafa:}
    
Ovaj graf pruža uvid u distribuciju trajanja pjesama. Na x-osi nalaze se različite razine trajanju u sekundama, dok y-os predstavlja broj pjesma u pojedinoj razini. Ovaj zanimljiv histogram omogućava vizualni o najčešćem trajanju pjesama.
    

    \textbf{Slika grafa:}
    \begin{figure}[H]
        \includegraphics[scale=0.9]{slike/Histogram of song durations.png}
        %veličina slike u odnosu na originalnu datoteku i pozicija slike
        \centering
        \caption{Histogram of song duration}
        
    \end{figure}


\eject



